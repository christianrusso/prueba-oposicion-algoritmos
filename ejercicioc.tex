\section{Ejercicio}
\subsection{Enunciado}
\frame{
\frametitle{Ejercicio 8) (Práctica 3 - M\'etodos Num\'ericos)}
Si A = $LL^t$ es una factorizaci\'on de A con L una matriz triangular inferior con elementos de la diagonal positivos. 
\newline
Demostrar que A es sim\'etrica y definida positiva.

}

\subsection{Camino a seguir}
\frame{

?`Qu\'e tenemos que probar?

\pause

\begin{itemize}
\item A es una matriz sim\'etrica 
\pause

\item A es una matriz definida positiva
\pause

\item Entonces, A es una matriz \textbf{sim\'etrica definida positiva}
\end{itemize}
}

\subsection{Solución}
\frame{
Primero veamos que A es una Matriz sim\'etrica. ?`C\'omo?

\begin{block}{Definici\'on }
Sea A $\in$ $\mathbb{R}^{nxn}$. A es \textbf{simetrica} sii 
\begin{center}
$a_{ij} = a_{ji}$ $\forall i,j $
\end{center}
\end{block}

\pause

Usando esta definicii\'on: 
\[  \left( \begin{array}{cccc}
  a_{11}   & \cdots &  a_{1n}  \\ 
  a_{21}   & \cdots &  a_{2n}  \\
 \vdots & \ddots  & \vdots \\
  a_{n1}   & \cdots &  a_{nn} 
\end{array} \right) =
\left( \begin{array}{cccc}
  l_{11} &  0 & \cdots &  0 \\ 
  l_{21} &  l_{22} & \cdots &  0 \\
 \vdots & \vdots & \ddots & \vdots \\
  l_{n1} &  l_{n2} & \cdots &  l_{nn}
\end{array} \right) 
\left( \begin{array}{cccc}
  l_{11} &  l_{21} & \cdots &  l_{n1} \\ 
  0 &  l_{22} & \cdots &  l_{n2} \\
 \vdots & \vdots & \ddots & \vdots \\
 0 & 0 & \cdots &  l_{nn}
\end{array} \right) \]
}

\frame{

En particular, podemos pensarla como  
\newline
\newline
\[ \left( \begin{array}{cccc}
 a_{11} & a_{12}  & a_{13} \\ 
 a_{21} & a_{22} & a_{23} \\
 a_{31} & a_{32} & a_{33}
\end{array} \right) 
 = LL^t=
\left( \begin{array}{cccc}
 l_{11} & 0  & 0 \\ 
 l_{21} & l_{22} & 0 \\
 l_{31} & l_{32} & l_{33}
\end{array} \right) 
\left( \begin{array}{cccc}
 l_{11} & l_{21}  & l_{31} \\ 
 0 & l_{22} & l_{32} \\
 0 & 0 & l_{33}
\end{array} \right) \]

\pause

multiplicando nos queda,

\begin{itemize}
\pause
\item $a_{11} = l_{11} * l_{11}$
\pause
\item $a_{12} = l_{11} * l_{21}$
\pause
\item $a_{13} = l_{11} * l_{31}$
\pause
\item $a_{21} = l_{21} * l_{11}$
\item $a_{22} = l_{21} * l_{21} + l_{22} * l_{22}$
\item $a_{23} = l_{21} * l_{31} + l_{22} * l_{32}$
\item $a_{31} = l_{31} * l_{11}$
\item $a_{32} = l_{31} * l_{21} + l_{32} * l_{22}$
\item $a_{33} = l_{31} * l_{31} + l_{32} * l_{32} + l_{33} * l_{33}$
\end{itemize}



}


\frame{
Ahora, queremos ver que $a_{ij} = a_{ji}$ $\forall i,j $

\begin{itemize}
\pause
\item $a_{12} = a_{21} = l_{11} * l_{21}$
\pause
\item $a_{13} = a_{31} = l_{11} * l_{31}$
\pause
\item $a_{23} = a_{32} = l_{21} * l_{31} + l_{22} * l_{32}$
\end{itemize}
\pause

Luego, por la definici\'on de matriz sim\'etrica, \textbf{A es sim\'etrica.}
}


\frame{
Ahora, nos queda probar que es definida positiva. ?`C\'omo?
\pause
\newline
Queremos ver que
\begin{center}
 $\forall x \neq 0$, $x^t A x > 0$
 \end{center}
 
 \pause
 
 Como $A = LL^t$ nos queda 

\begin{center}
 $\forall x \neq 0$, $ x^t * L*L^t *x > 0$
 \end{center}
 
 \pause
 \textbf{Recordando:}
 \begin{itemize}
 \item  $\left\| v \right\| $ $\geq$ 0
 \item  $\left\| v \right\| $ $\neq$ 0 sii v $\neq$ 0
 \end{itemize}

\pause
 
 Volviendo, tratemos de usar estas propiedades de la norma.
 }

\frame{


\begin{center}
 $\forall x \neq 0$, $ x^t * L*L^t *x > 0$
 \end{center}
 \pause
 
 \begin{center}
 $\forall x \neq 0$, $(x^t L)(x^t L)^t > 0$
 \end{center}
 
 \pause
 \begin{center}
 $\forall x \neq 0$, $\left\| L^t * x \right\| ^2 > 0$
 \end{center}
  \pause
  
 Veamos que $L^t * x \neq 0$  ?`C\'omo?  
 
\begin{block}{Propiedad }
Sea A una matriz inversible. Entonces la ecuaci\'on $Ax = 0$ tiene como \'unica soluci\'on x = 0
\end{block}

 
 \pause

Probemos que $L^t$ es inversible.
\pause
Dado que los elementos de la diagonal son positivos tenemos que

\[ det(L) = \prod_{i = 1}^n l_{ii}  > 0 \]



Luego, A es inversible. 
\pause
Entonces \textbf{A es sim\'etrica definida positiva}.

}



